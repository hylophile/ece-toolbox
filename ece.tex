\documentclass[10pt,landscape]{article}
\usepackage[american]{circuitikz}
\usepackage{amsmath}
\begin{document}
\footnotesize
% \begin{multicols}{3}

\newcommand{\bcz}{\begin{circuitikz}}
\newcommand{\ecz}{\end{circuitikz}}


\begin{center}
     \Large{\textbf{Intro to ECE Toolbox}} 
\end{center}

\section{Ohm's law and some basics}
\subsection{Questions}
What's the voltage at $X$? What's the current through $Y$?
What's the resistance of $Z$? How much power is used by $F$?
\subsection{Ohm's law}
$V=I \cdot R$ \hspace{1cm}
$I=\frac{V}{R}$ \hspace{1cm}
$R=\frac{V}{I}$ \hspace{1cm}
$P=RI^2=VI$ \vspace{1cm}

\bcz[american voltages] % how to show both the arrow and plus / minus?
  \draw
  (0,0) to [R,v>=$$] (2,0);
\ecz
\subsection{Current Division}
\bcz % add symbols
  \draw
  (0,0) to [short] (4,0)
  (4,0) to [R] (4,2)
  (2,0) to [R] (2,2)
  (4,2) to [short] (0,2);
%  \hspace{1cm}
\ecz
$I_k=I_s\left( \frac{\frac{1}{R_k}}{\sum\limits_i{\frac{1}{R_i}}} \right)$
\subsection{Voltage Division}
\bcz[american voltages] % add symbols
  \draw
  (0,0) to [short, o-] (1,0)
  (1,0) to [R] (1,2)
  (1,2) to [R=$R_k$,v_>=$V_k$] (1,4)
  (1,4) to [short, -o] (0,4);
\ecz
$V_k=V_s\left( \frac{R_k}{\sum\limits_i R_i} \right)$
\subsection{Resistor Reduction}
\paragraph{Series}
%\begin{align} % tried aligning horizontally... nope not today
\bcz
  \draw
  (0,0) to [R,R=$R_1$] (2,0)
  (2,0) to [R,R=$R_2$] (4,0);
\ecz
=
\bcz
  \draw
  (0,0) to [R,R=$R_{eq}$] (2,0);
\ecz
$R_{eq}=R_1+R_2+\dots$
%\end{align}
\paragraph{Parallel} 
\bcz % add symbols
  \draw
  (0,0) to [short] (4,0)
  (4,0) to [R=$R_2$] (4,2)
  (2,0) to [R=$R_1$] (2,2)
  (4,2) to [short] (0,2);
\ecz
$=$
\bcz
  \draw
  (0,0) to [R,R=$R_{eq}$] (2,0);
\ecz
$R_{eq}=\left( \frac{1}{R_1}+\frac{1}{R_2}+\dots \right)^{-1}$
\subsection{Component Order}
\paragraph{Series}
\bcz
  \draw
  (0,0) to [R] (2,0)
  (2,0) to [V<=] (4,0); % minus is rotated ;/
\ecz
=
\bcz
  \draw
  (0,0) to [V<=] (2,0) % same here
  (2,0) to [R] (4,0);
\ecz
\paragraph{Parallel} 
\bcz 
  \draw
  (0,0) to [short] (4,0)
  (4,0) to [R] (4,2)
  (2,2) to [I] (2,0)
  (4,2) to [short] (0,2);
\ecz
$=$
\bcz 
  \draw
  (0,0) to [short] (4,0)
  (4,2) to [I] (4,0)
  (2,2) to [R] (2,0)
  (4,2) to [short] (0,2);
\ecz
% first page kinda done!


% \newcommand{\mitem}[2]{#1 & #2 \\}
% \begin{tabular}{@{}ll@{}}
%     \mitem{Pellentesque}{Eget nisl ut lorem fringilla elementum.}
%     \mitem{Curabitur}{Consequat nisi at ligula hendrerit condimentum.}
%     \hline
%     \mitem{Pellentesque}{Eget nisl ut lorem fringilla elementum.}
%     \mitem{Curabitur}{Consequat nisi at ligula hendrerit condimentum.}
% \end{tabular}

% \end{multicols}
\end{document}
