\documentclass[10pt,landscape]{article}
\usepackage{circuitikz}
\begin{document}
\footnotesize
% \begin{multicols}{3}

\begin{center}
     \Large{\textbf{Intro to ECE Toolbox}} 
\end{center}

\section{Ohm's law and some basics}
\subsection{Questions}
What's the voltage at $X$? What's the current through $Y$?
What's the resistance of $Z$? How much power is used by $F$?
\subsection{Ohm's law}
$V=I \cdot R$ \hspace{1cm}
$I=\frac{V}{R}$ \hspace{1cm}
$R=\frac{V}{I}$ \hspace{1cm}
$P=RI^2=VI$ \vspace{1cm}

\begin{circuitikz}[american voltages] % how to show both the arrow and plus / minus?
  \draw
  (0,0) to [R,v>=$$] (2,0);
\end{circuitikz}
\subsection{Current Division}
\begin{circuitikz} % add symbols
  \draw
  (0,0) to [short] (4,0)
  (4,0) to [R] (4,2)
  (2,0) to [R] (2,2)
  (4,2) to [short] (0,2);
%  \hspace{1cm}
\end{circuitikz}
$I_k=I_s\left( \frac{\frac{1}{R_k}}{\sum\limits_i{\frac{1}{R_i}}} \right)$
\subsection{Voltage Division}
\begin{circuitikz} % add symbols
  \draw
  (0,0) to [short] (1,0)
  (1,0) to [R] (1,2)
  (1,2) to [R] (1,4)
  (1,4) to [short] (0,4);
\end{circuitikz}
\subsection{Resistor Reduction}

\subsection{Component Order}




% \newcommand{\mitem}[2]{#1 & #2 \\}
% \begin{tabular}{@{}ll@{}}
%     \mitem{Pellentesque}{Eget nisl ut lorem fringilla elementum.}
%     \mitem{Curabitur}{Consequat nisi at ligula hendrerit condimentum.}
%     \hline
%     \mitem{Pellentesque}{Eget nisl ut lorem fringilla elementum.}
%     \mitem{Curabitur}{Consequat nisi at ligula hendrerit condimentum.}
% \end{tabular}

% \end{multicols}
\end{document}
